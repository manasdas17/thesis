\prefacesection{Abstract}
The ionosphere plays an important role in the geospace system, perturbing and disrupting radio signals that pass through it and coupling the Earth with its space environment. Of particular note to spacecraft designers and operators are the small meteoroids that routinely impact spacecraft and are observable due to their passage through the ionosphere. Mechanical and electrical damage due to hypervelocity impacts with meteoroids pose a significant threat to spacecraft, and to fully characterize the threat it is necessary to observe the sporadic meteoroids through the meteor plasma that they create in the ionosphere. These observations, in addition to those of ionospheric plasmas that disrupt communications, can be made with high-power large-aperture radars located across the globe. Unfortunately, current radar techniques are not sufficient for elucidating the details of these plasmas due to inherent signal processing artifacts.

Sparsity has recently gained prominence in many scientific fields as a means of speeding, simplifying, or increasing the accuracy of information processing. Often this research falls under the moniker of \emph{compressed sensing}, which exploits sparsity, global measurement, and nonlinear reconstruction to acquire data faster or with higher resolution. Mathematically, signals like images, sound waves, or radar returns can be represented with vectors. The vectors are paired with a dictionary describing atoms that, when combined according to the vector elements, can represent any given signal. Working with vectors gives a concrete definition of sparsity; it means that most of a vector's elements are zero. Natural phenomena tend to have structure, and that structure manifests itself as sparsity in the signal's vector representation with an appropriate dictionary. Knowing that natural signals are sparse allows one to constrain solutions to underdetermined measurements and uniquely find the signal's true representation.

In order to enable flexible high-resolution measurements of ionospheric plasma phenomena, a sparsity-based radar waveform inversion technique is formulated and found to eliminate processing artifacts caused by the standard matched filter approach. Taking direction from the theory of compressed sensing, sparsity of the radar target scene is employed as prior knowledge to successfully perform the inversion. The result is cleaner data that limits self-interference of range-spread targets and enables differentiation in crowded and variable environments. Though the approach has been applied to ionospheric radar, it is generally applicable and especially relevant for radar target scenes with multiple or distributed scatterers.

As a basis for the technique, a discrete radar model that captures signal sparsity in a delay-frequency dictionary is developed. This model is shown to have a strong connection to existing methods, resulting in an intuitive interpretation of the inversion technique as an iterative thresholding matched filter. An explicit formulation of the discrete model's representation of arbitrary distributed scatterers is derived, and it shows that sparsity is reasonably preserved in the discrete representation. Building on top of the model, waveform inversion is implemented using modern convex optimization techniques tailored for efficient computation and quick convergence. Finally, the real-world flexibility and effectiveness of the inversion technique is demonstrated by the elimination of filtering artifacts from meteor observations made with a variety of standard radar waveforms.

\newpage
